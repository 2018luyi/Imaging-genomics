\documentclass[9pt]{article}

\usepackage[legalpaper, margin=1in]{geometry}
\usepackage{graphicx}

\begin{document}
\section{Summary of papers on imaging genetics}

\subsection{Review on multivariate methods in imaging genetics, by Liu and Calhoun (2014)}

The authors distinguish \textbf{four categories} of multivariate methods, depending on the dimensionality of the phenotype and genotype investigated:
\begin{enumerate}
  \item \textbf{Univariate} analysis of candidate variants on candidate phenotypes.
  \item \textbf{Multivariate} analyses on genetic data. They can be \textbf{\emph{a priori}-based} or \textbf{data-driven}.
  \item Multivariate analysis on imaging data.
  \item Multivariate analysis on both modalities.
\end{enumerate}

The paper reviews methods from categories 2 to 4.

\subsubsection{Category 2:}
Data-driven methods:
\begin{itemize}
\item\textbf{Multidimensional reduction} (MDR) is an attribute construction algorithm that creates a new variable by pooling genotypes.
\item \textbf{Principal component analysis} (PCA):  generates a set of orthogonal components.
\item \textbf{Independent Component Analysis} (ICA): generates a set of statistically independent components. See next section for details.
\item \textbf{Clustering} methods.
\end{itemize}

\subsubsection{Category 3}
\begin{itemize}
\item\textbf{ICA}: consider an observed $M-$dimensional random vector $x=(x_1, x_2, ..., x_M)$ which is generated by the ICA model $X=AS$ where $S$ is the source matrix. The goal of ICA is to estimate an unmixing matrix $W$ such that $Y=WX$ is a good approximate to the "true" sources.
\item\textbf{Independent vector analysis} (IVA): generalization of ICA for analysis of multiple datasets.
\end{itemize}


\subsubsection{Category 4: Multivariate analyses bridging imaging and genetics.}

\begin{equation}
X=(x_1, x_2, ..., x_p)^t,\, Y=(y_1, y_2, ..., y_q)^t
\end{equation}

$X$ is $n\times p$, $Y$ is $n\times q$, where $n$ is the number of individuals, $p$ is the number of SNPs and $q$ is the number of voxels.	
\begin{itemize}
\item CCA (canonical component analysis [sic]): connects \emph{two sets of variables}, $X$ and $Y$, by finding maximally correlated linear combinations of the variables. Let us call these linear combinations $U:=AX$ and $V:=BY$.
\item PLS: idem previous, but maximizes the covariance between the latent variables.
\item RRR (reduced rank regression):
\item ICA (independent component analysis):
\end{itemize}


\subsection{Multi-omics factor analysis, by Argelaguet \emph{et al.}}

\begin{figure}[h!]
  \includegraphics[width=\linewidth]{figs/mofa}
  \label{fig:mofa}
\end{figure}


\subsection{Detection of relationships among multi-modal brain imaging
meta-features via information flow, by Miller \emph{et al.}}

\begin{figure}[h!]
  \includegraphics[width=\linewidth]{figs/miller}
  \label{fig:cond_prob}
\end{figure}


\subsection{A three-way parallel ICA approach to analyze links among genetics, brain structure and brain function}

\end{document}